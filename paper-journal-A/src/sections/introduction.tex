\section{Introduction}
Reliable heartbeat classification is pivotal for early arrhythmia detection, triage of acute coronary syndromes, and longitudinal monitoring of chronic cardiovascular disease. Automated analysis can extend specialist-grade interpretation to ambulatory settings where cardiologists are scarce, supporting telehealth and wearable deployments that respond to surging arrhythmia prevalence worldwide\cite{hannun2019cardiologist,attia2019artificial,ribeiro2020automatic,gladis2024ecg,huillcen2025efficient}. Despite decades of progress, false alarms and missed detections remain commonplace because irregular heartbeats exhibit subtle morphologies, patient-specific variability, and severe class imbalance that challenges conventional classifiers\cite{acharya2017deep,chawla2002smote,he2009learning,saito2015precision}.

Classical ECG analysis pipelines combined handcrafted features with probabilistic or rule-based decision logic. Landmark contributions such as the MIT-BIH Arrhythmia Database curation\cite{moody2001impact} and robust QRS detection algorithms\cite{pan1985real,laguna1994automatic} laid the foundation for patient-adapted heartbeat labelling\cite{dechazal2004patient} and margin-based classification\cite{osowski2001support}. Subsequent studies enriched the corpus with wavelet and morphological descriptors to mitigate noise sensitivity while preserving clinical interpretability\cite{addison2005wavelet,torrence1998practical,mallat1989wavelet}.

The past decade has seen deep learning approaches eclipse many of these handcrafted pipelines by leveraging large ECG corpora and end-to-end optimisation. Convolutional, recurrent, and hybrid models have demonstrated strong accuracy on ambulatory and wearable ECG datasets\cite{luz2016ecg,kiranyaz2015personalized,zubair2016automated,yildirim2018arrhythmia,oh2018automated,li2018atrial}. Modern residual and attention-based architectures further enhance representation learning for 12-lead and long-duration monitoring scenarios\cite{zheng2020residual,faust2018deep,ismailfawaz2019deeplearning}. Yet, many deployments still rely on large parameter counts or bespoke feature engineering, motivating lightweight designs that retain robustness to patient heterogeneity.

Recent advances in deep sequence modelling motivate hybrid architectures that fuse convolutional inductive biases with global attention. Conformer encoders---originally proposed for speech recognition---combine depthwise convolutions, self-attention, and macaron-style feed-forward blocks to capture both local and long-range dynamics in biosignal time series\cite{gulati2020conformer}. We hypothesised that adapting this architecture to Morlet wavelet representations of electrocardiograms (ECGs) would improve discrimination across the American Association of Medical Instrumentation (AAMI) heartbeat classes compared with purely convolutional or recurrent baselines.

The present study evaluates this hypothesis on the benchmark MIT-BIH Arrhythmia Database, a curated corpus of ambulatory ECG records widely used for arrhythmia algorithm validation\cite{goldberger2000physiobank}. Building on the open-source pipeline released with this work, we (i) document a reproducible data preparation and streaming framework, (ii) quantify performance gains delivered by a lightweight Conformer network relative to attention-only, CNN-LSTM, and feature-based baselines, and (iii) analyse clinical implications, limitations, and future extensions needed to translate the approach into practice.
