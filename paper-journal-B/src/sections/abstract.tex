Timely arrhythmia triage remains limited by scarce electrophysiology expertise, motivating explainable ECG classifiers that can operate across ambulatory data streams. We used the MIT-BIH Arrhythmia Database comprising 48 half-hour, two-lead recordings with 110,919 annotated beats across the AAMI categories and prepared Morlet wavelet scalograms from 187-sample windows around each R-peak, aggregated into three-beat sequences to preserve temporal context and clinician-readable morphology. A CNN-Conformer model with macaron feed-forward layers, relative self-attention, and depthwise convolution was trained with class-weighted cross-entropy using five-fold cross-validation prior to evaluation on a stratified held-out test cohort, alongside attention-only and CNN-LSTM baselines. The final system achieved 60\,% accuracy, macro-F1 0.26, and weighted-F1 0.68 on the test set, with mean one-vs-rest AUC 0.66 (0.90 for normal beats) and validation accuracy 0.44\,±\,0.16, outperforming the strongest baseline by 33 percentage points in accuracy and 0.11 in macro-F1. These results indicate that interpretable time--frequency features paired with lightweight Conformer inductive biases deliver reproducible gains for heartbeat classification while exposing residual limitations on rare fusion beats, supporting deployment as decision support for telecardiology workflows and guiding future work on data augmentation and clinician-in-the-loop calibration.
