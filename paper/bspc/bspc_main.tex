% BSPC main file (elsarticle)
\documentclass[preprint,3p]{elsarticle}
% Preamble for Journal of Electrocardiology submissions using elsarticle
\usepackage[margin=1in]{geometry}
\usepackage{setspace}
\usepackage{graphicx}
\usepackage{siunitx}
\usepackage{hyperref}
\usepackage{booktabs}
\usepackage{amsmath}
\usepackage{enumitem}
\usepackage{xcolor}
\usepackage{tikz}
\usetikzlibrary{arrows.meta,positioning,shapes.geometric,shapes.symbols}
\usepackage{csquotes}

\doublespacing

\hypersetup{
  colorlinks=true,
  linkcolor=blue,
  citecolor=blue,
  urlcolor=blue
}

% Ensure figures/tables appear near their references.
\setlength{\textfloatsep}{12pt}
\setlength{\floatsep}{12pt}


\begin{document}
\begin{frontmatter}
\title{Heartbeat Classification with Conformer-CNN on MIT-BIH: AAMI-oriented Evaluation with Robustness and Calibration}
\author[1]{Alireza Abbaszadeh\corref{cor1}}\ead{alireza.abbaszadeh.research@gmail.com}
\author[2]{Armita Shahlaee}\ead{Armita.shahlaee@gmail.com}
\cortext[cor1]{Corresponding author.}
\address[1]{Department of Computer Engineering, Ma.C., Islamic Azad University, Mashhad, Iran}
\address[2]{Department of Biological Sciences and Technologies, Faculty of Basic Sciences, Islamic Azad University, Mashhad, Iran}

\begin{abstract}
We present a practical ECG heartbeat classifier trained and evaluated on \dataset{}, following the \aami{} scheme (N/S/V/F/Q). The method leverages spectro-temporal representations and a lightweight Conformer-CNN, and we report comprehensive metrics (macro-F1, PR-AUC, ROC-AUC) alongside calibration and robustness analyses. All results and figures are harvested from the public repository logs and artifacts for strict reproducibility.\todo{Tighten to BSPC word/page guidance later.}
\end{abstract}

\begin{keyword}
ECG; Arrhythmia; MIT-BIH; AAMI; Conformer; Calibration; Robustness; Reproducibility
\end{keyword}
\end{frontmatter}

\section{Introduction}
\label{sec:intro}
\todo{Cite clinical motivation; summarize imbalance challenges; contributions list.}

\section{Related Work}
\label{sec:related}
\todo{Deep ECG classification, AAMI protocols, calibration in medical ML.}

\section{Methods}
\label{sec:methods}
\subsection{Data and Preprocessing}
We use \dataset{} with AAMI mapping to five classes (N/S/V/F/Q).\todo{Add class counts from auto table.}
\subsection{Model}
Conformer-CNN hybrid with spectro-temporal front-end.\todo{Add architecture table/diagram.}
\subsection{Training and Evaluation}
Inter-patient split; metrics: accuracy, macro-F1, ROC-AUC, PR-AUC; calibration: ECE/Brier.\todo{Hyperparameters.}

\section{Results}
\label{sec:results}
\subsection{Overall performance}
% Leaderboard across runs (auto-generated)
\begin{table}[H]\centering
\caption{Leaderboard across available runs (best/median/worst by macro-F1).}
\label{tab:leaderboard}
\begin{tabular}{lrr}
\toprule
Run & Acc & Macro-F1 \\ 
\midrule
-- & -- & -- \\ 
\bottomrule
\end{tabular}
\end{table}

\subsection{Per-class curves and confusion}
\begin{figure}[H]\centering\includegraphics[width=.48\textwidth]{../figures/confusion_matrix.png}\includegraphics[width=.48\textwidth]{../figures/precision_recall.png}\caption{Confusion matrix and PR curves.}\label{fig:conf_pr}\end{figure}
\begin{figure}[H]\centering\includegraphics[width=.6\textwidth]{../figures/roc_curves.png}\caption{ROC curves by class.}\label{fig:roc}\end{figure}
\subsection{Calibration and robustness}
% Auto-fill when probability outputs are availabing inference and update this table.
\endgroup

% Robustness under noise/perturbations
\begin{table}[H]\centering
\caption{Robustness under synthetic noise (SNR) or perturbations.}
\label{tab:robustness}
\begin{tabular}{lrrrr}
\toprule
Perturbation & Level & Acc & Macro-F1 & Notes \\ 
\midrule
Gaussian & -- & -- & -- & -- \\ 
Baseline wander & -- & -- & -- & -- \\ 
EM noise & -- & -- & -- & -- \\ 
\bottomrule
\end{tabular}
\end{table}


\section{Discussion}
\label{sec:discussion}
\todo{Clinical implications, error modes in S/F classes, limitations.}

\section{Reproducibility}
All scripts and figures are harvested from the repository at commit \commit{TBD}. The script `scripts/copy_core_figs.py` copies PNGs.\todo{Pin exact SHA.}

\section{Conclusion}
\label{sec:conclusion}
\todo{Summarize gains, practical readiness, and future work.}

\bibliography{../references}
\end{document}
