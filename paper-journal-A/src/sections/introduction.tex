\section{Introduction}
Reliable heartbeat classification is pivotal for early arrhythmia detection, triage of acute coronary syndromes, and longitudinal monitoring of chronic cardiovascular disease. Automated analysis can extend specialist-grade interpretation to ambulatory settings where cardiologists are scarce, supporting telehealth and wearable deployments that respond to surging arrhythmia prevalence worldwide\cite{hannun2019cardiologist}. Despite decades of progress, false alarms and missed detections remain commonplace because irregular heartbeats exhibit subtle morphologies, patient-specific variability, and severe class imbalance that challenges conventional classifiers\cite{acharya2017deep}.

Recent advances in deep sequence modelling motivate hybrid architectures that fuse convolutional inductive biases with global attention. Conformer encoders---originally proposed for speech recognition---combine depthwise convolutions, self-attention, and macaron-style feed-forward blocks to capture both local and long-range dynamics in biosignal time series\cite{gulati2020conformer}. We hypothesised that adapting this architecture to Morlet wavelet representations of electrocardiograms (ECGs) would improve discrimination across the American Association of Medical Instrumentation (AAMI) heartbeat classes compared with purely convolutional or recurrent baselines.

The present study evaluates this hypothesis on the benchmark MIT-BIH Arrhythmia Database, a curated corpus of ambulatory ECG records widely used for arrhythmia algorithm validation\cite{goldberger2000physiobank}. Building on the open-source pipeline released with this work, we (i) document a reproducible data preparation and streaming framework, (ii) quantify performance gains delivered by a lightweight Conformer network relative to attention-only, CNN-LSTM, and feature-based baselines, and (iii) analyse clinical implications, limitations, and future extensions needed to translate the approach into practice.
