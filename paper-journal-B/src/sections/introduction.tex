\section{Introduction}
Cardiac arrhythmias remain a leading cause of morbidity worldwide, yet diagnostic capacity outside tertiary centers is limited by access to electrophysiology expertise and the logistical burden of manual electrocardiogram (ECG) review. Contemporary deep learning systems have demonstrated the feasibility of classifying dozens of rhythm phenotypes from ambulatory recordings, but clinical adoption still hinges on robustness across noisy wearable data streams and the transparency of the generated explanations.\cite{Hannun2019Cardiologist,Gadaleta2023SingleLead,Kim2024FortyFive,Ozpolat2023Quantum} These challenges are magnified in telecardiology pathways where automated screening must triage escalating data volumes without increasing false alarms that erode clinician trust or delay interventions.

Open cardiac signal repositories such as the MIT-BIH Arrhythmia Database have catalyzed reproducible benchmarking while seeding advances in interpretable artificial intelligence that align with regulatory expectations for decision support tools.\cite{Goldberger2000PhysioBank,Gliner2025Interpretability,Czerwinski2025Interpretable} Nevertheless, state-of-the-art methods often emphasize predictive accuracy over clinician-facing explanations, leaving a gap in evidence for architectures that can natively surface waveform rationales and propagate uncertainty through multi-class predictions.

Parallel progress across biomedical engineering demonstrates how interpretable modeling, multi-modal sensing, and artifact-aware preprocessing can expand the fidelity of automated monitoring systems. For example, BioMedical Engineering OnLine studies have showcased transparent classifiers for complex motion analytics, hybrid artifact rejection strategies for ballistic cardiography, and segmentation frameworks that preserve physiological context in infant monitoring and pressure-distribution mapping.\cite{Yoon2023Interpretable,Jiang2025Hybrid,Voss2023Multimodal,Stern2024InBed} These advances underscore the importance of combining human-readable representations with resilient signal pipelines when translating algorithms into safety-critical environments.

Building on these insights, we investigate a convolutional-Conformer network trained on time--frequency representations of MIT-BIH heartbeats with the dual goals of improving macro-averaged discrimination and delivering token-level attention maps tailored for electrophysiology review. By synthesizing multi-scale processing blocks with calibration-aware training, the present study contributes an end-to-end workflow that addresses prevalent pain points reported in recent BioMedical Engineering OnLine imaging and gait analysis research, while situating model interpretability and economic considerations within the broader AI-enabled ECG literature.\cite{Chen2024SuperResolution,Qiu2025Attention,Sugiarto2025GroundReaction,Hsieh2025Economic,Lee2024Diastolic}
