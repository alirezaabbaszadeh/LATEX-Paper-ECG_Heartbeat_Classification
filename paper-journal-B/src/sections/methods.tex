\section{Methods}
\subsection{Data sources and cohort construction}
We curated heartbeats from the MIT-BIH Arrhythmia Database, selecting the 48 half-hour, two-lead Holter recordings and harmonizing annotations under the AAMI EC57 taxonomy to enable comparison with recent open benchmarks.\cite{Goldberger2000PhysioBank,Kim2024FortyFive} Each record was partitioned into non-overlapping training, validation, and test folds at the subject level to prevent patient leakage, yielding 110,919 labeled beats after discarding corrupted segments and rhythm transitions that lacked consensus annotations. To reflect telemonitoring workloads, we preserved class imbalance rather than oversampling rare supraventricular or fusion beats, and we reserved 20\% of each arrhythmia subtype for the held-out evaluation cohort described in \autoref{sec:results}.

\subsection{Signal conditioning and representation learning}
Raw ECG leads were band-pass filtered between 0.5 and 40~Hz before R-peak localization via a wavelet-based detector, after which 187-sample windows centered on each fiducial were extracted and concatenated into three-beat sequences. Inspired by BioMedical Engineering OnLine investigations into interpretable motion analytics and artifact suppression, we generated complex Morlet wavelet scalograms to retain morphology while dampening motion-induced jitter.\cite{Jiang2025Hybrid,Yoon2023Interpretable,Voss2023Multimodal} The two-lead scalograms were z-scored per record, stacked as pseudo-RGB images, and augmented with small temporal warps, Gaussian noise, and polarity inversions to emulate wearable noise while respecting clinical plausibility.

\subsection{Neural architecture and training procedure}
The classifier combined a shallow convolutional stem with a two-layer Conformer encoder comprising macaron feed-forward blocks, multi-head self-attention with relative positional encoding, and depthwise convolution modules for local context refinement. Dropout (0.2) and stochastic depth (0.1) regularized the network, while layer normalization stabilized gradient flow. Architectural choices were informed by BioMedical Engineering OnLine image-segmentation work highlighting the benefits of residual attention and multi-scale context, alongside locomotor modeling studies that integrated anatomical priors into deep regressors.\cite{Chen2024SuperResolution,Qiu2025Attention,Sugiarto2025GroundReaction} We optimized class-weighted cross-entropy with the AdamW optimizer (initial learning rate $1\times10^{-4}$, weight decay $5\times10^{-3}$) using cosine annealing over 120 epochs and mini-batches of 256 examples on NVIDIA A100 GPUs. Early stopping monitored macro-averaged F1 on the validation fold with a patience of 15 epochs.

\subsection{Baselines and ablation strategy}
To contextualize performance, we reproduced two strong baselines: a convolutional recurrent network following the CNN-LSTM specification in recent ambulatory arrhythmia studies and an attention-only Transformer tuned to match the Conformer parameter count.\cite{Fira2025SingleChannel,Kirkbas2025CommonMatrix} Both baselines ingested identical scalograms and were trained with the same optimizer schedule. Additionally, we conducted ablations removing the depthwise convolution branch, substituting mel-spectrogram inputs, and disabling the class-weighted loss to quantify the effect of each design decision on minority class recall.

\subsection{Evaluation metrics and statistical analysis}
Model selection prioritized macro-averaged F1 to reflect balanced sensitivity across all arrhythmia types, while weighted F1, overall accuracy, and one-vs-rest area under the receiver operating characteristic curve (AUROC) provided complementary views of performance. Confidence intervals for each metric were estimated via 1,000 bootstrap resamples of the test set with patient-level blocks to preserve dependency structure.\cite{Hempel2025ECGAging,Hsieh2025Economic} McNemar tests compared paired classification outcomes between the Conformer and baseline models, with Bonferroni correction applied to control the family-wise error rate across rhythm categories. Saliency heat maps were generated from the attention weights and aggregated per class to support qualitative inspection in downstream clinical workflows.
