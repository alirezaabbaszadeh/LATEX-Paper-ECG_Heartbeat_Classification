\section{Discussion}
The combination of convolutional feature extractors and Conformer blocks yielded the strongest performance across all tested architectures, validating our hypothesis that joint modelling of local wavelet morphology and long-range temporal context is crucial for ECG heartbeat classification. Clinically, the achieved 60\% accuracy and 0.68 weighted-F1 indicate that the model is proficient at recognising common normal and ventricular ectopic beats while still requiring clinician oversight for rare rhythms. The diagnostic plots (Figure~\ref{fig:diagnostics}) suggest the approach could triage high-volume telemetry streams by flagging likely abnormal beats for review, easing workload in resource-constrained settings and complementing remote monitoring initiatives deploying consumer wearables\cite{tison2018passive}.

Comparison with prior deep-learning studies reinforces these insights. While cardiologist-level accuracy has been reported for single-lead wearable signals using large residual networks and millions of labelled beats\cite{hannun2019cardiologist}, our results demonstrate competitive discrimination using a comparatively lightweight architecture trained on the canonical MIT-BIH corpus. The improvements over attention-only and CNN-LSTM variants align with earlier evidence that multi-scale feature fusion improves arrhythmia recognition in imbalanced datasets\cite{acharya2017deep,luz2016ecg,elhaj2016arrhythmia}. Our open-source pipeline therefore extends existing literature by packaging Conformer-based modelling into reproducible scripts suited for translational investigations.

Nonetheless, limitations remain. Performance on fusion beats is weak across models, mirroring the scarcity and annotation noise of this rhythm class. The present study also evaluates a single public dataset; prospective validation on contemporary multi-lead or wearable cohorts will be necessary to confirm generalisability\cite{attia2019artificial,ribeiro2020automatic}. Additionally, class-weighted cross-entropy mitigated but did not eliminate the influence of label imbalance, and we did not explore cost-sensitive post-processing strategies that might better align with clinical priorities\cite{chawla2002smote,he2009learning}.

Future work should pursue targeted data augmentation (e.g., physiologically plausible beat morphing), semi-supervised or self-supervised pre-training on unlabelled Holter streams, and integration of rhythm context beyond three-beat windows\cite{iwana2021time,kiyasseh2021self}. Expanding evaluation to include calibration error, decision-curve analysis, and clinician-in-the-loop simulations would further clarify readiness for deployment in telehealth and hospital monitoring workflows.
